%% get rid of autoindent   :setl noai nocin nosi inde=
%%
%% 1. pdflatex main
%% 2. bibtex main
%% 3. pdflatex main
%% 3. pdflatex main
%\documentclass[review]{elsarticle}
\documentclass[]{article}
%\documentclass[preprint]{elsarticle}

\usepackage{lineno,hyperref}
\modulolinenumbers[5]

\usepackage[fleqn]{amsmath} 
\usepackage{amssymb}
\usepackage{graphicx}
\usepackage{tabularx} 
\usepackage{footnote}
\usepackage{titlesec}
\titleformat{\section}{\large\bfseries}{\thesection}{1em}{}

\makesavenoteenv{tabular}
\makesavenoteenv{table}
\usepackage[percent]{overpic}
\usepackage{subfigure}% subcaptions for subfigures
\usepackage{subfigmat}% matrices of similar subfigures, aka small multiples
\usepackage{array}
\setlength{\mathindent}{0pt}

\renewcommand{\labelitemi}{$\triangleright$}
\renewcommand{\labelitemii}{$\triangleright$}
\renewcommand{\labelitemiii}{$\triangleright$}
\renewcommand{\labelitemiv}{$\triangleright$}


%\journal{Journal of Computational Physics}

\newcommand{\mb}{\mathbf}
\newcommand{\tn}{\textnormal}
\newcommand{\unit}[1]{\ensuremath{\, \mathrm{#1}}}

\newcommand{\bmz}{\mbox{\boldmath $z$\unboldmath}}
\newcommand{\bmr}{\mbox{\boldmath $r$\unboldmath}}
\newcommand{\bmb}{\mbox{\boldmath $b$\unboldmath}}
\newcommand{\bms}{\mbox{\boldmath $s$\unboldmath}}
\newcommand{\bmg}{\mbox{\boldmath $g$\unboldmath}}
\newcommand{\bmq}{\mbox{\boldmath $q$\unboldmath}}
\newcommand{\bmU}{\mbox{\boldmath $U$\unboldmath}}
\newcommand{\bmu}{\mbox{\boldmath $u$\unboldmath}}
\newcommand{\bmv}{\mbox{\boldmath $v$\unboldmath}}
\newcommand{\bmw}{\mbox{\boldmath $w$\unboldmath}}
\newcommand{\bmm}{\mbox{\boldmath $m$\unboldmath}}
\newcommand{\bmi}{\mbox{\boldmath $i$\unboldmath}}
\newcommand{\bmj}{\mbox{\boldmath $j$\unboldmath}}
\newcommand{\bmk}{\mbox{\boldmath $k$\unboldmath}}
\newcommand{\bmx}{\mbox{\boldmath $x$\unboldmath}}
\newcommand{\bmX}{\mbox{\boldmath $X$\unboldmath}}
\newcommand{\bmH}{\mbox{\boldmath $H$\unboldmath}}
\newcommand{\bmy}{\mbox{\boldmath $y$\unboldmath}}
\newcommand{\bmn}{\mbox{\boldmath $n$\unboldmath}}
\newcommand{\bmf}{\mbox{\boldmath $f$\unboldmath}}
\newcommand{\bmF}{\mbox{\boldmath $F$\unboldmath}}
\newcommand{\bmS}{\mbox{\boldmath $S$\unboldmath}}
\newcommand{\bmG}{\mbox{\boldmath $G$\unboldmath}}
\newcommand{\bmV}{\mbox{\boldmath $V$\unboldmath}}
\newcommand{\bmI}{\mbox{\boldmath $I$\unboldmath}}
\newcommand{\bmD}{\mbox{\boldmath $D$\unboldmath}}
\newcommand{\bmnu}{\mbox{\boldmath $\nu$\unboldmath}}

\newcommand{\DefTen}{\mathbb{D}}
\newcommand{\EyeTen}{\mathbb{I}}

\DeclareMathOperator{\Ei}{Ei}

\newcommand{\MVBcmnt}[1]{{\color{red}{\em #1}}}

\newcolumntype{L}[1]{>{\raggedright\let\newline\\\arraybackslash\hspace{0pt}}m{#1}}
\newcolumntype{C}[1]{>{\centering\let\newline\\\arraybackslash\hspace{0pt}}m{#1}}
\newcolumntype{R}[1]{>{\raggedleft\let\newline\\\arraybackslash\hspace{0pt}}m{#1}}

%%\bibliographystyle{elsarticle-num}
\bibliographystyle{acm}
%%%%%%%%%%%%%%%%%%%%%%%

\title{A non-intrusive numerical method for determing the most
  dangerous unstable modes of discretized
  multiphase/multimaterial systems.
  \thanks{this material is based upon work supported by National
   Aeronautics and Space Administration under grant number
   80NSSC20K0352}}

%%!!format authors properly
%\author{
%	LastName1, FirstName1\\
%	\texttt{first1.last1@xxxxx.com}
%	\and
%	LastName2, FirstName2\\
%	\texttt{first2.last2@xxxxx.com}
%}
\author{
  Okojunu, Abel \\
  Dept. of math, FL State Univ. 
  \and
  Nair, Unnikrishnan Sasidharan \\
  Dept. of Mech. Engr., FL State Univ.
  \and
  Sussman, Mark \\
  Dept. of math, FL State Univ. 
}

\begin{document}
\maketitle
%% Xinv [x1 x2 ... xn]=[e1 e2 .... en]
%% u_t = f(u)
%% suppose u0 is a steady state => f(u0)=0
%% now solve u_t=f(u0)+grad_u f(u0) (u-u0) in which u(0)=u0+du
%% let A=grad_u f(u0)
%% u_t=f(u0)+A(u-u0)  u(0)=u0+du 
%% since f(u0)=0 =>
%% u_t=A(u-u0) u(0)=u0+du
%% suppose AX=X Lambda =>
%% u_t=X Lambda Xinv (u-u0)
%% let V=Xinv u =>
%% V_t=Lambda (V-V0) => vk=e^{\lambda_k t}dvk+v0k
%% suppose du is the kth eigenvector of A times eps =>
%% du is the kth column of X times eps=>
%% dv=Xinv du=eps ek
%% consider time discrete problem:
%% Vnp1-Vn=dt Lambda (V-V0) ....
%% Vn=v0+eps(1+lambda dt)^n
ABSTRACT:
A linear stability analysis technique is presented for finding the most
unstable modes for a discretized multiphase/multi material system.  
Instead of explicitly linearizing a discretized system of ODEs and then
investigating the spectrum of the linearized equations, it is proposed
to do a ``matrix-free'' power method approach.  In the proposed
linear stability analysis approach, the largest (in magnitude)
eigenvalue of the linearized discritized equations is found.
The new 
power method approach determines the action of the linear operator, 
associated with the linearized equations, on a given eigenfunction by 
way of strategically perturbing a base state and then applying the
original, unmodified, multiphase flow algorithm.  
The multiphase flow algorithm is the coupled level set and 
``continuous MOF'' method for multiphase/multimaterial problems.  Since the
underlying multiphase flow solver is an Eulerian interface capturing approach, 
solutions with complex topology can be recovered automatically without 
interface ``surgery.'' Furthermore, since the proposed linear
stability algorithm does not require modification of the
underlying flow solver, then it is possible to find unstable modes for
complex topology problems.  Examples are include for Rayleigh capillary
instability and a Jet in Cross Flow problem.



% Fred Stern -> reconstructed distance function coupled with VOF 
%%cite: Mukundan
\linenumbers
\section{Introduction}
We present an improved ``Continuous Moment-of-Fluid'' (CMOF) algorithm
for computing solutions to multiphase (multimaterial) flows.  We 
demonstrate the benefits of our new algorithm, beyond that of the 
current state-of-the-art, on these benchmark multiphase flow
problems:
bubble formation\cite{helsby1955behaviour} (see Section \ref{bubform}),
freezing\cite{hu2010icing} (see Section \ref{freezing_sec}), 
liquid lens\cite{MIAO2021109358} (see Section \ref{liqlens}),
and bubble dynamics in a Cryogenic fuel 
tank\cite{bentz1993low} (see Section \ref{TPCEsec}).

Figure \ref{checkerboard} (LEFT figure) 
illustrates the 
Moment-of-Fluid\cite{dyadechko2005moment,ahn2007multi,ahn2009adaptive}
reconstruction of a ``saw tooth'' function that has a wave
length of $\Delta x$.  The MOF reconstruction is very ``noisy'' in the
sense that the highest frequency
Fourier coefficients of the discrete Fourier transform
of the reconstruction do not decay with repeated reconstructions.
In other-words the amplitudes of the high frequency Fourier modes may
not decay when exposed to the repeated process of
(1) MOF advection, and (2) MOF reconstruction.
The presence of $O(\Delta x)$ wave length noise is
a problem for applications involving  
surface tension driven flows.  Standard
Volume-of-Fluid techniques for extracting the curvature
from volume fractions\cite{sussman2003second,cummins2005estimating}
will extract a zero curvature from a ``saw tooth'' interface 
since the associated volume fraction field {\em and} centroid
field varies in the $y$
direction only.  In other words, the
underlying computational fluid dynamics surface tension force
algorithm will never ``see'' the jagged interface and therefore will have
no way for removing the noise.  If there is viscosity, the presence of the 
noise is unphysical and the noise can lead to collateral damage (see Figure 
\ref{MOF_liquid_lens} and Table \ref{liquidlens_table}) 
to the overall flow field; the noise can
result in loss of accuracy.

In order to overcome the ``MOF checkerboard instability'' issue, we have 
developed the ``Continuous Moment of Fluid'' (CMOF) interface reconstruction
algorithm\cite{VAHAB2021}.  
Referring to Figure \ref{checkerboard}, the repeated process
of (1) CMOF advection, followed by (2) CMOF reconstruction, 
will quickly eliminate the
noise.  The difference between CMOF and MOF is that in the CMOF algorithm,
the reference centroid is the material centroid relative to the 
encompassing $3\times 3\times 3$ stencil of grids cells, rather than just the
center cell.  See Figure \ref{CMOFcentroid}.   

We remark that alternative approaches to using MOF for multiphase 
(multimaterial)
flows, e.g the level set 
method\cite{smith2002projection,shetabivash2020multiple,starinshak2014new}, 
phase field method\cite{HUANG2022110795}, 
front tracking method \cite{vu2015numerical},
or standard ``PLIC'' VOF 
methods\cite{ancellin2022extension,schofield2008material,schofield2009second}, 
do not have the
``MOF checkerboard instability'' problem.  These alternative approaches
have other difficulties.  The level set or front tracking
multimaterial 
approaches\cite{smith2002projection,shetabivash2020multiple,starinshak2014new,vu2015numerical} 
are not volume preserving methods.  
The phase field method\cite{HUANG2022110795} 
smears the interface over 3 grid
cells or more.  The standard, second order, ``PLIC'' VOF methods for 
$M$ material multimaterial 
flows\cite{ancellin2022extension,schofield2008material,schofield2009second} 
minimize a cost function which has $27(M-1)$ 
degrees of freedom; i.e. the control variable space 
is $27(M-1)$ dimensions.

Our improved CMOF method, on the other-hand, admits a sharp reconstructed
interface, is volume preserving, and the control variable space has only
$3(M-1)$ dimensions.  The CMOF reconstruction algorithm has been 
improved in this article by applying ``Decision Tree'' 
machine learning techniques 
in order to rapidly determine the optimal ``CMOF'' slope.
We choose the Decision Tree algorithm\cite{breiman1984classification} which is 
a ``lossless'' method if one chooses to store all of the training
data samples in the tree structure.  
In Table \ref{tab:triple_point}, we summarize the
existing state-of-the-art for numerical methods for 
multiphase (multimaterial)
flows with surface tension.

\begin{table}[htbp]
  \centering
  \scalebox{0.7}
  {
  \begin{tabular}[h]{|C{2cm}|C{2cm}|C{2cm}|C{2cm}|C{2cm}|C{2cm}|C{2cm}|}
    \hline
    Author(s) & 
    Triple point\newline reconstruction\newline algorithm\newline DOF$^{1}$ &
    Coupled\newline with fluid &
    Volume\newline preserving &
    Sharp\newline interface & 
    Curvature\newline discretization \\
    \hline
    Smith et al.~\cite{smith2002projection} & Level set & Yes & No  & Yes & Level set \\ \hline
    Ahn and Shahkov\cite{ahn2007multi} & $\tn{VOF-GRAD}^2$\newline $\tn{VOF-LVIRA}^3$ \newline MOF 
                 \newline 27 or 3 $\times(M-1)$& No & Yes & Yes & N/A\\ \hline
    Kim\cite{kim2007phase} & Phase field & Yes & Yes & No & Phase field \\ \hline
    Dyadechko and Shashkov\cite{dyadechko2008reconstruction} & MOF \newline $3(M-1)$ & No & Yes & Yes & N/A \\ \hline
    Caboussat et al.~\cite{caboussat2008numerical} & $\tn{VOF-IP}^4$ \newline $27(M-1)$ & Yes & No & Yes & 
                                           Convolution/Height function\cite{francois2006balanced}\\ \hline
    Schofield et al.~\cite{schofield2008material, schofield2009second} & $\tn{VOF-PD}^5$ \newline $27(M-1)$ 
	                                         & No & Yes & Yes & N/A \\ \hline
    Sijoy and Chaturvedi\cite{sijoy2010volume} & $\tn{VOF-PLIC}^6$ \newline $27(M-1)$& 
	                                      Yes & Yes & Yes & N/A \\ \hline
    Kucharik et al.~\cite{kucharik2010comparative} & VOF-PLIC \newline VOF-PD \newline MOF \newline 27 or 3 $\times(M-1)$ &
	                                         Yes & Yes & Yes & N/A \\ \hline
    Bonhomme et al.~\cite{bonhomme2012inertial} & VOF \newline $27(M-1)$ & Yes & Yes & No & VOF \\ \hline
    Starinshak et al.~\cite{starinshak2014new} & Interface level set & Yes & No & Yes & N/A \\ \hline
    Vu et al.~\cite{vu2015numerical} & Front tracking & Yes & No & Yes & Front tracking \\ \hline
    Pathak and Raessi\cite{pathak2016three} & VOF-PLIC \newline $27(M-1)$ & No & Yes & Yes & N/A \\ \hline
    Shetabivash et al.~\cite{shetabivash2020multiple} & Level Set & Yes & No & Yes & Finite Difference \\ \hline
    Ancellin et al.~\cite{ancellin2022extension} & VOF-PLIC \newline $27(M-1)$ & No & Yes & Yes & N/A \\ \hline
    Huang et al.~\cite{HUANG2022110795} & Phase field & Yes & Yes & No & Phase field \\ \hline
    Present Article, Ye et al & CMOF$^7$ \newline $3(M-1)$ & Yes & Yes & Yes & VOF Height function \\ \hline
  \end{tabular}
  }
 \scalebox{0.7}
 {
 \begin{minipage}{\textwidth}
   $^1$: Degrees of Freedom for 3D reconstruction; $M$ is the number of materials \\
   $^2$: Gradient based interface reconstruction\\
   $^3$: Least squares volume-of-fluid interface reconstruction algorithm\\
   $^4$: Interior-point method for the localization of the triple point\\
   $^5$: Power diagram\\
   $^6$: Piecewise linear interface construction\\
   $^7$: Continuous Moment of Fluid construction\\
   \end{minipage}
 }

\caption{Recent numerical methods of flows with triple points in 
  chronological order.  \label{tab:triple_point} }
\end{table}


\section{Background - deforming boundary problems in fluid mechanics}

In the Introduction, we briefly demonstrate the benefits of our
new method over the current state-of-the-art numerical methods
for multi-phase flows in which surface tension forces are important.  
In this section, we give an overview of numerical methods for 
deforming boundary problems in computational fluid dynamics in general,
and show where our new method fits in.

For many deforming boundary problems in fluid dynamics, ``shock capturing''
numerical methods are 
sufficient\cite{godunov1959finite,colella1984piecewise,van1979towards,harten1997high,shu1988efficient,saurel1999multiphase}. Shock capturing methods
have limitations.  According to 
``Godunov's theorem''\cite{godunov1954different}, linear one-step second
order accurate numerical methods for $\phi_{t}+c\phi_{x}=0$ cannot be
monotonicity preserving unless $|c|\Delta t/\Delta x$ is an 
integer.  The concept of ``monotonicity preserving'' methods and its
association to TVD (Total Variational Diminishing) methods was first 
drawn by Harten\cite{harten1997high} who proved that (a) A monotone
scheme is TVD and (b) a TVD scheme is monotonicity preserving.  
When Harten's theory is taken together with Godunov's theorem, one can
see that ``linear'' second order TVD methods only exist under
specialized situations.  Harten's and Godunov's theory are manifested by the
fact that stable shock capturing methods will invariably smear 
discontinuities over time.  

In order to overcome the limitations of shock capturing methods, researchers
have developed (i) front tracking 
methods\cite{glimm1981front,unverdi1992front},
(ii) front capturing 
methods\cite{markstein1951interaction,osher1988fronts,sussman1994level,hirt1981volume,brackbill1992continuum,ahn2007multi,ahn2009adaptive,olsson2005conservative,QiuETALPINNPHASEFIELD2022,
huang2020consistent}, and
(iii) shock fitting methods\cite{SalasShockfitting1976}.

The above methods for overcoming the
the time dependent 
``interface smearing'' problem of the shock capturing methods, have been
further improved by way of hybridization:
the particle level set method\cite{enright2002hybrid},
the hybrid front tracking and level set method\cite{shin2009hybrid},
the coupled level set and volume of fluid method\cite{sussman2000coupled},
and the coupled level set and moment of fluid 
method\cite{ASURIMUKUNDAN2022110864,jemison2013coupled}.

The numerical method that we introduce in this article falls in the latter
``hybrid'' category in which we introduce the hybrid level set and 
Continuous Moment of Fluid (CMOF) method for multiphase/multimaterial
flows.  The level set method and the continuous moment-of-fluid method
are hybridized as follows:
(a) the smooth level set distance function representation is used
to provide an initial guess for the CMOF slope reconstruction
step (see Figure \ref{distancefunction} and Section \ref{MOF_vs_CMOF_sec}),
and (b) the level set distance functions are in turn replaced by the
exact signed distance to the continuous moment-of-fluid reconstructed
interface (see Figure \ref{distancefunction} and 
Section \ref{MMRECON}).  The level set and CMOF representations are 
synchronized every time step.


\section{Mathematical model}
We simulate the flow of a multiphase system consisting of $M_{fluid}$
fluid (deforming) materials and $M_{rigid}$ non-deforming materials.
The $M_{fluid}$ materials tessellate the computational domain and in the
case that a fluid material $m_{fluid}$ coincides with a rigid material,
$m_{rigid}$, the rigid materials' governing equations take precedence.
Otherwise, in regions where fluid materials and rigid materials do not
overlap, the fluid materials are
governed by the incompressible Navier-Stokes equations of immiscible
flows.   We refer the reader to Figures \ref{tank_levelset},
\ref{ice_levelset}, and \ref{liquid_lens} for illustrations which 
distinguish between deforming (tessellating) materials and non-deforming
(rigid) materials.
\begin{itemize}
\item \textbf{Material domain and interface:}
Mathematically, the domain of rigid \emph{material} $m_{rigid}$ is the 
region in which $\phi_{m_{rigid}}>0$:
\begin{equation}
  \phi_{m_{rigid}}(\mb{x},t) = \left\{
    \begin{array}{ll}
      > 0 & \mb{x} \in \tn{material}\,\: m_{rigid}, \\ 
      \leq 0 & \tn{otherwise},
  \end{array}
\right.
\end{equation}
$\bmx$ is the position vector in space and $t$ is time. 
The domain of \emph{material} $m_{fluid}$ is the 
region in which $\phi_{m_{fluid}}>0$ and 
$\phi_{m_{rigid}}<0$:
\begin{equation}
  \phi_{m_{fluid}}(\mb{x},t) = \left\{
    \begin{array}{ll}
      > 0 & \mb{x} \in \tn{material}\,\: m_{fluid}\cup m_{fluid,ghost}, \\ 
      \leq 0 & \tn{otherwise},
  \end{array}
\right.
\end{equation}

The \emph{interface} level set,
$\phi_{m1,m2}$
represents the interface between materials
$m1$ and $m2$
\begin{equation}
  \phi_{m1,m2}(\mb{x},t) = \left\{
    \begin{array}{ll}
      > 0 & \mb{x} \in \tn{material}\,\: m1, \\ 
      < 0 & \mb{x} \in \tn{material}\,\: m2, \\ 
      = 0 & \mb{x}\,\: \tn{along}\,\:(m1,m2)\,\:\tn{interface}. \\ 
    \end{array}
\right.
\end{equation}
The normal and curvature defined based on these level set functions are:
\begin{equation}
  \mb{n}_{m1,m2} = 
   \frac{\nabla \phi_{m1,m2}}{|\nabla \phi_{m1,m2}|}, \hspace{10pt}   
   \kappa_{m1,m2} = 
   \nabla \cdot \frac{\nabla \phi_{m1,m2}}{|\nabla \phi_{m1,m2}|}.
\end{equation}
\item \textbf{Conservation of mass:}
We assume that each fluid material, $m_{fluid}$, is incompressible, 
so that the velocity 
field $\bmu = (u, v, w)$ is divergence free within the bulk of each
fluid material:
\begin{equation}
  \label{eq:cnt}
  \nabla \cdot \bmu = 0.
\end{equation}
In order to account for phase change or otherwise sources and sinks of mass,
we have the following conditions on $\nabla \cdot \bmu$:
\begin{eqnarray*}
\nabla \cdot \bmu = 
  \sum_{\mbox{sources}} 
  \frac{\dot{m}_{\mbox{source}}}
       {\rho_{\mbox{source}}}\delta(\phi_{m_{\mbox{source}}}) -
  \sum_{\mbox{sinks}} 
  \frac{\dot{m}_{\mbox{sink}}}
       {\rho_{\mbox{sink}}}\delta(\phi_{m_{\mbox{sink}}}) 
\end{eqnarray*}

$H(\phi)$ is the Heaviside function which is defined as,
\begin{eqnarray*}
	H(\phi)=\left\{ \begin{array}{cc}
		1 & \phi>0  \\
  		0 & \phi\le 0 \end{array} 
	\right.
\end{eqnarray*}
$\delta(\phi)$ is the Dirac Delta function,
\begin{eqnarray*}
	\delta(\phi)=H'(\phi).
\end{eqnarray*}
%units of k: watts/(m Kelvin)=kg m/(s^3 Kelvin)
%units of L: J/kg
%1 WATT=1 Joule/s=kg m^2/s^3
%1 Joule=kg m^2/s^2
%k grad T/L units=kg m/(s^3 Kelvin)  *  (Kelvin/m)  * (kg/J)=
%kg^2/(s^3 J)=kg^2/(s kg m^2)=kg/(m^2 s)
For boiling examples,
$\dot{m}$ is the mass flux of boiling 
liquid across the liquid/vapor interface,
% e.g. liquid on left, vapor on right
% \bmn=-1
% grad T_{l}<0
% grad T_{v}>0
% \dot{m}>0
\begin{eqnarray*}
  \dot{m} = 
  \frac{k_{l}\nabla T_{l}\cdot \bmn_{l,v} - 
        k_{v}\nabla T_{v}\cdot \bmn_{l,v}}{L},
\end{eqnarray*}
where 
$k_{l}$ and $k_{v}$ are the thermal conductivities in the 
liquid and ambient vapor regions respectively, $\rho_{l}$ and
$\rho_{v}$ are the densities in the liquid and ambient vapor
regions respectively, $L$ is the latent heat of vaporization,
and $\bmn_{l,v}$ is the interface normal vector pointing from the
ambient vapor region into the liquid,
\begin{eqnarray*}
	\bmn_{l,v} = \frac{\nabla\phi_{l,v}}{|\nabla\phi_{l,v}|}.
\end{eqnarray*}


\item \textbf{Conservation of momentum:}
The conservation of momentum for each material in its domain is given by
\begin{eqnarray}
 (\bmu\rho_{m})_{t}+
  \nabla\cdot(\bmu \otimes \bmu \rho_{m}+p_{m} \EyeTen)=
  \label{eq:NS} \\
  \nabla\cdot(2\mu_{m}\DefTen)+
  \rho_{m}\bmg (1-\alpha_{m}(T_{m}-T_{0m}))
  \hspace{10pt} \tn{if} \hspace{10pt} \phi_{m}(\bmx,t)>0,
  \nonumber
\end{eqnarray}
where $p_{m}$, $T_{m}$, $\alpha_{m}$, and $\mu_{m}$ are 
pressure, temperature, coefficient of thermal expansion, and 
viscosity of material $m$ respectively, 
$\mb{g}$ is the gravitational acceleration vector, and 
$\DefTen= (\nabla\bmu+(\nabla\bmu)^{T})/2$ is the rate of deformation tensor. 
\item \textbf{Conservation of energy:} 
The conservation of energy for each material in its domain is given by
\begin{equation}
  \label{eq:heat}
  (\rho_{m}C_{p,m}T_{m})_t+
  \nabla\cdot(\bmu\rho_{m}C_{p,m}T_{m})=
  \nabla\cdot ( k_{m}\nabla T_{m})
  \hspace{10pt} \tn{if} \hspace{10pt} \phi_{m}(\bmx,t)>0,
\end{equation}
where $C_{p,m}$ and $k_{m}$ are heat capacity and thermal conductivity of 
material $m$ respectively, and $T_{m}$ is the temperature. 
\item \textbf{Interfacial jump condition:}
Here, we write out the general equations for a deforming
interface changing phase.  We define $m_{s}$ and $m_{d}$ as
the material id's associated with a ``source'' material 
(e.g. boiling liquid or freezing liquid)
and ``destination'' material (e.g. vapor from boiling or ice from freezing) 
respectively.
The location of the interface separating a material $m_{s}$ region from a
material $m_{d}$ region is governed by the level set equation,
\begin{eqnarray}
\phi_{m_{s},m_{d},t} + 
\bmu_{m_{s}}\cdot\nabla\phi_{m_{s},m_{d},t} =
-\frac{\dot{m}}{\rho_{s}}|\nabla\phi_{m_{s},m_{d},t}|
\end{eqnarray}
\begin{eqnarray}
\phi_{m_{d},m_{s},t} + 
\bmu_{m_{s}}\cdot\nabla\phi_{m_{d},m_{s},t} =
\frac{\dot{m}}{\rho_{s}}|\nabla\phi_{m_{d},m_{s},t}|
\end{eqnarray}
An equivalent expression for the level set governing equations is:
\begin{eqnarray}
\phi_{m_{s},m_{d},t} + 
\bmu_{m_{d}}\cdot\nabla\phi_{m_{s},m_{d},t} =
-\frac{\dot{m}}{\rho_{d}}|\nabla\phi_{m_{s},m_{d},t}|
\end{eqnarray}
\begin{eqnarray}
\phi_{m_{d},m_{s},t} + 
\bmu_{m_{d}}\cdot\nabla\phi_{m_{d},m_{s},t} =
\frac{\dot{m}}{\rho_{d}}|\nabla\phi_{m_{d},m_{s},t}|
\end{eqnarray}
The interface jump conditions,
between two materials $m_{s}$ and $m_{d}$,
for the velocity, pressure, 
and temperature, respectively, are,
\begin{eqnarray*}
 \bmu_{m_{s}}\cdot \bmn_{m_{s},m_{d}}-
 \bmu_{m_{d}}\cdot \bmn_{m_{s},m_{d}} = 
\dot{m}(\frac{1}{\rho_{m_{d}}}-
	\frac{1}{\rho_{m_{s}}}),
\end{eqnarray*}
\begin{eqnarray*}
	(p_{m_{s}}\EyeTen-
	 p_{m_{d}}\EyeTen).\bmn_{m_{s},m_{d}}=
	-\sigma_{m_{s},m_{d}}\kappa_{m_{s},m_{d}}\bmn_{m_{s},m_{d}} + \\
	(2\mu_{m_{s}}\DefTen_{m_{s}}-
	 2\mu_{m_{d}}\DefTen_{m_{d}}).\bmn_{m_{s},m_{d}},
\end{eqnarray*}
\begin{eqnarray}
	T_{m_{s}}=
	T_{m_{d}}.
	\label{temperature_jump}
\end{eqnarray}
$\kappa_{m_{s},m_{d}}$ is the interface curvature
and is defined as,
\begin{eqnarray*}
	\kappa_{m_{s},m_{d}}=\nabla\cdot
	\frac{\nabla\phi_{m1,m2}}
             {|\nabla\phi_{m1,m2}|}.
\end{eqnarray*}

At a triple point junction a three-phase equilibrium known as the 
Neumann's triangle\cite{de2013capillarity} determines the 
angles (see Figure \ref{fig:triple_point}.a):
\begin{equation}
  \label{eq:triple_point_eqlib}
  \frac{\sin(\theta_{1})}{\sigma_{23}}=
  \frac{\sin(\theta_{2})}{\sigma_{13}}=
  \frac{\sin(\theta_{3})}{\sigma_{12}}.
\end{equation}

\end{itemize}


\section{Numerical methods and algorithms}

We describe our numerical method for the 
2D uniform rectangular Cartesian grid
case.  
In the problem domain, a computational cell, $\Omega_{i,j}$, is defined as,
\begin{eqnarray}
  \label{eq:cell_def}
  \Omega_{i,j} = 
   \left\{ \bmx: x \in \left[x_i-\frac{\Delta x}{2}, x_i+\frac{\Delta x}{2}\right],\right.\\
   \left. y \in \left[ y_i-\frac{\Delta y}{2}, y_i+\frac{\Delta y}{2}\right] \right\}
\end{eqnarray}
where $\bmx_{i,j} = \left\{x_i,y_j\right\}$ is the 
center of the cell $\Omega_{i,j}$. The domain of 
material $m$ in a cell at time $t^n$ is denoted by $\Omega_{m,i,j}^n$, 
and the zeroth and first order moments of the $m$th material 
distribution, corresponding to the volume fraction and 
centroid position, are defined as,
\begin{equation}
  \label{eq:moments_def}
  F_{m,i,j}^n = \frac{\int_{\Omega_{m,i,j}^n} ~\mathrm{d} \Omega}{V_{i,j}}, \hspace{0.5cm}\bmx_{m,i,j}^{c,n} = \frac{\int_{\Omega_{m,i,j}^n} \bmx ~\mathrm{d} \Omega}{V_{i,j,m}^n},
\end{equation}
where the computational cell volume is $V_{i,j} = \int_{\Omega_{i,j}} ~\mathrm{d} \Omega$, and volume of the portion for material $m$ in a computational cell is $V_{i,j,m}^n= \int_{\Omega_{m,i,j}^n} ~\mathrm{d} \Omega$. The discretization for a typical point around the triple point is shown in Figure \ref{fig:triple_point}.

\subsection{method overview: Staggered grid Projection Method}

Referring to Figure \ref{fig:ch2_nodeslocation} and also 
\cite{pei2019hierarchical,VAHAB2021}, 
we discretize the velocity on the marker-and-cell (MAC) grid and we
discretize pressure, temperature, level set function(s), volume fraction(s),
and centroid(s) at the center of grid cells.

An outline of our operator split method is as 
follows 
(\cite{VAHAB2021}):
\begin{itemize}
\item \textbf{Calculate the time step $\Delta t=t^{n+1}-t^{n}$:}
  \begin{eqnarray*}
  \Delta t_{1}=\mbox{min}_{d=1,\ldots,\mbox{D}}
    \frac{\mbox{min}\Delta x_{d}}{2 D \mbox{max}|u_{d}^{n}|} 
    \hspace{15pt} \mbox{D}=2\mbox{ or }3
  \end{eqnarray*}
  \begin{eqnarray*}
  \Delta t_{2}=
    \mbox{min}_{d=1,\ldots,\mbox{D}}
    \frac{\mbox{min}\Delta x_{d}}{D\mbox{max}
    |u_{d}^{\mbox{phase change},n-1}|} 
  \end{eqnarray*}
  \begin{eqnarray}
  \Delta t_{3}=
  \mbox{min}_{d=1,\ldots,D}
  \mbox{min}_{m1,m2=1,\ldots,M}
    \Delta x_{d}^{3/2}
    \sqrt{\frac{\rho_{m1}+\rho_{m2}}
          {2\pi\sigma_{m1,m2}}} 
    \label{stiffconstraint}
  \end{eqnarray}
  \begin{eqnarray*}
  \Delta t=\mbox{CFL}\mbox{min}(
    \Delta t_{1},
    \Delta t_{2},
    \Delta t_{3}) \hspace{15pt} \mbox{CFL}=1/2
  \end{eqnarray*}
  
\item \textbf{Cell Integrated Semi-Lagrangian (CISL) MAC velocity advection
  (see\cite{pei2019hierarchical}):}
  \begin{eqnarray}
  (\rho \bmu)_{t}+\nabla\cdot(\rho\bmu\bmu)=0, \label{advection}
  \end{eqnarray}
\item \textbf{CISL temperature advection
  (liquid and gas materials, see \cite{VAHAB2021}):}
  \begin{eqnarray}
  (\rho_{m} C_{p,m} T_{m})_{t}+
  \nabla\cdot(\rho_{m} C_{p,m} \bmu T_{m})=0, \hspace{10pt}
  m=1,\ldots,M \label{adv2}
  \end{eqnarray}
\item \textbf{CISL level set advection:}
  \begin{eqnarray*}
  \phi_{m,t}+\bmu\cdot\nabla\phi_{m}=0, \hspace{10pt} 
  m=1,\ldots,M 
  \end{eqnarray*}
\item \textbf{CISL advection of the Volume fractions and Centroids
 (see \cite{jemison2013coupled,jemison2014compressible,li2015multiphase,pei2019hierarchical,VAHAB2021}):}
  \begin{eqnarray}
  F_{m,t}+\nabla\cdot(\bmu F_{m})=0, \hspace{10pt}
  m=1,\ldots,M  \label{eq:transport}
  \end{eqnarray}
\item \textbf{Coupling the level set functions to the volume fractions and Centroids:}
  The (continuous) 
  moment-of-fluid reconstructed
  (see Section \ref{MOF_vs_CMOF_sec}) 
  slope initial guess uses the level set functions,
  and the level set functions are in turn replaced by the exact 
  signed distance (see Section \ref{MMRECON}) 
  to the continuous moment-of-fluid reconstructed
  interface.
\item \textbf{Phase change velocity from material $m_{s}$ (source) to material $m_{d}$ (destination)
  (see \cite{VAHAB2021} and \cite{liu2022novel}):}
  \begin{eqnarray}
  \bmu^{\mbox{phase change}}=-\frac{\dot{m}}{\rho_{m_{d}}}\bmn_{m_{d}},
  \label{uphasechange}
  \end{eqnarray}
\item \textbf{Phase change: Level Set Advection:}
  \begin{eqnarray*}
  \phi_{m,t}+\bmu^{\mbox{phase change}}\cdot\nabla\phi_{m}=0,
  \hspace{10pt} m=m_{s} \mbox{ or } m_{d}.
  \end{eqnarray*}
\item \textbf{Phase change: unsplit CISL advection of the Volume fractions and Centroids
  (see \cite{VAHAB2021,liu2022novel}):}
  \begin{eqnarray*}
  F_{m,t}+\nabla\cdot(\bmu^{\mbox{phase change}}F_{m})=0,
  \hspace{10pt} m=m_{s} \mbox{ or } m_{d}.
  \end{eqnarray*}
\item \textbf{Phase change: Coupling the level set functions to the volume fractions and Centroids:} 
  The (continuous) 
  moment-of-fluid reconstructed
  (see Section \ref{MOF_vs_CMOF_sec}) 
  slope initial guess uses the level set functions,
  and the level set functions are in turn replaced by the exact 
  signed distance (see Section \ref{MMRECON}) 
  to the continuous moment-of-fluid reconstructed
  interface.
\item \textbf{Phase change: Mass source redistribution:} 
  Redistribute $\dot{m}$ to the
  source material, $m_{s}$, side (e.g. liquid if boiling or 
  freezing) \cite{VAHAB2021}.
\item \textbf{Thermal diffusion
  (see \cite{VAHAB2021} for algorithmic details):}
  \begin{eqnarray}
  \frac{(\rho C_{p,m})^{\mbox{mix},n+1}}{\Delta t^{\mbox{swept}}}
  (T^{n+1}_{m}-T^{\mbox{advection}}_{m})=
   \nabla\cdot (k_{m} \nabla T^{n+1}_{m})
  \label{diffusion1}
  \end{eqnarray}
\item \textbf{Viscosity
  (see \cite{pei2019hierarchical} for algorithmic details):}
  \begin{eqnarray}
	  \frac{\rho^{\mbox{mix},n+1}}{\Delta t}
	  (\bmu^{\ast}-\bmu^{\mbox{advection}})=
	  \nabla\cdot(2\mu\DefTen^{\ast})-
	  \rho^{n+1}(\alpha^{n+1}(T^{n+1}-T_{0}))\bmg
	  \label{diffusion3}
  \end{eqnarray}
\item \textbf{Pressure Gradient (liquid and vapor regions)
  (see 
  \cite{jemison2014compressible,li2015multiphase,pei2019hierarchical,VAHAB2021} 
  for algorithmic details):}
  \begin{eqnarray}
  \frac{\bmu^{n+1}-\bmu^{\ast}}{\Delta t}=
    -\frac{\nabla p^{n+1}}{\rho^{\mbox{MAC,mix},n+1}}+\bmg-
    \frac{\sum_{m=1}^{M} \gamma_{m}\kappa_{m}\nabla H(\phi_{m})}
         {\rho^{\mbox{MAC,mix},n+1}}
	  \label{pressuregrad}
  \end{eqnarray}
  \begin{eqnarray*}
  \nabla \cdot \bmu^{n+1} = 
   \sum_{\mbox{sources}} 
   \frac{\dot{m}_{\mbox{source}}}
        {\rho_{\mbox{source}}}\delta(\phi_{m_{\mbox{source}}}) -
   \sum_{\mbox{sinks}} 
   \frac{\dot{m}_{\mbox{sink}}}
        {\rho_{\mbox{sink}}}\delta(\phi_{m_{\mbox{sink}}}) 
  \end{eqnarray*}
  For the case in which two materials, $m1$ and $m2$,
  are present in a given $3\times 3$
  stencil,
   %% remember: \kappa_{m1}=-\kappa_{m2}, H(\phi_{m1})=1-H(\phi_{m2})
  \begin{eqnarray*}
  \gamma_{m1}=\gamma_{m2}=\frac{\sigma_{m1,m2}}{2} 
  \end{eqnarray*}
  and for the case when three materials, $m1$, $m2$, $m3$, are
  present in a given $3\times 3$ stencil,
  \begin{eqnarray*}
  \gamma_{m1}=\frac{\sigma_{m1,m2}+\sigma_{m1,m3}-\sigma_{m2,m3}}{2} 
  \end{eqnarray*}
  \begin{eqnarray*}
  \gamma_{m2}=\frac{\sigma_{m1,m2}+\sigma_{m2,m3}-\sigma_{m1,m3}}{2} 
  \end{eqnarray*}
  \begin{eqnarray*}
  \gamma_{m3}=\frac{\sigma_{m1,m3}+\sigma_{m2,m3}-\sigma_{m1,m2}}{2} 
  \end{eqnarray*}

\end{itemize}



Remarks:
\begin{enumerate}
\item For step 2 ``phase change'' above, we use MOF interface
 reconstruction.   Our rationale is that (1) the phase change
 velocity $\bmu^{\mbox{phase change}}$ is not  
 determined from the interface curvature, so that there is no
 instability risk due to MOF advection here, and (2) MOF is
 more accurate than CMOF in this scenario.
\item We use the multigrid preconditioned conjugate gradient 
 method \cite{tatebe1993multigrid}
 to solve the large sparse matrix system that
 results from discretizing (\ref{diffusion1}).  The discretization
 of the Dirichlet interface temperature condition due to 
 phase change uses the second order method described in 
 \cite{gibou2002second}.
\item We also use the multigrid preconditioned conjugate gradient (MGPCG)
 method \cite{tatebe1993multigrid}
 to solve the large sparse matrix system that
 results from discretizing (\ref{pressuregrad}).  In order
 to simulate flows in sealed tanks or flows induced by the 
 sealing of a valve \cite{arienti2014embedded} we have developed a general 
 method for enforcing the solvability condition in each 
 ``enclosed'' region.  We identify each ``enclosed'' region
 using a shading method developed by \cite{sussman2009stable}.
\item In order to preserve mass for phase change problems, we
  must prescribe a stringent tolerance for the pressure equation
  (\ref{pressuregrad}).  In some cases, due to round off error, the
  MGPCG method might stall.  In order to overcome the effect of 
  round-off error, we keep track of a history of the pressure
  admitting the least residual, and, when the solver has stalled,
  we restart the MGPCG process using the
  previous best guess pressure and
  increasing the number of multigrid relaxation steps by 1.
\end{enumerate}

\subsection{MOF and Continuous MOF interface reconstruction methods
  \label{MOF_vs_CMOF_sec} }
The original MOF reconstruction method is local to the cell and 
uses the reference volume fraction, $F_\tn{ref} \equiv F_{m,i,j}^n$, 
and reference centroid, $\bmx_\tn{ref}^c \equiv \bmx_{m,i,j}^n$ to 
find the linear (planer) interface reconstruction that has the volume
fraction equal to $F_\tn{ref}$, and has the least amount of error 
for centroid position. We find the actual volume fraction 
$F_\tn{act}(\bmn,b)$ and centroid $\bmx_\tn{act}^c(\bmn,b)$ 
for a reconstructed line(plane) with the normal $\bmn$ and 
intercept $b$ which minimizes
\begin{equation}
  E_\tn{MOF} = \lVert \bmx_\tn{ref}^c - \bmx_\tn{act}^c(\bmn,b) \rVert_2
\end{equation}
while $F_\tn{act}(\bmn,b) = F_\tn{ref}$ (see Figure
\ref{fig:mof_reconstruction}).

Starting from a whole cell and repeating this process 
while only considering the uncaptured regions from previous 
steps, we can reconstruct the material interface for 
each phase in a multimaterial cell (see \cite{li2015multiphase} 
for algorithm details). This tessellating procedure 
generates a volume preserving reconstruction 
at triple-points (see Figure \ref{fig:mof_tes_nontes}).

The Continuous Moment-of-fluid method (CMOF) employs a similar 
procedure to find the interface reconstruction in a cell, 
but uses a different value for the reference cell 
centroid $\bmx_\tn{ref}^c$. We define a super cell 
\begin{eqnarray}
  \label{eq:super_cell_def}
  \Omega_{i,j}^s = 
\left\{ \bmx: x \in \left[x_i-\frac{3\Delta x}{2}, x_i+\frac{3\Delta x}{2}\right],\right.\\
\left. y \in \left[ y_i-\frac{3\Delta y}{2}, y_i+\frac{3\Delta y}{2}\right] \right\}
\end{eqnarray}
with volume fraction and centroid
\begin{equation}
\label{eq:super_cell_f_x}
F_{m,i,j}^{s,n} = \frac{\sum_{i,j} F_{m,i,j}^n V_{i,j}}{\sum_{i,j} V_{i,j}}, 
\hspace{0.25cm}
\bmx_{m,i,j}^{c,s,n} = 
\frac{\sum_{i,j} F_{m,i,j}^n \bmx_{m,i,j}^{c,n} V_{i,j}}
     {\sum_{i,j} F_{m,i,j}^n V_{i,j}}.
\end{equation}
For the CMOF method we find the slope $\bmn$ and intercept $b$ such that 
$F_\tn{act}(\bmn,b) = F_\tn{ref}$, 
$E_\tn{MOF} = \lVert \bmx_\tn{ref}^c - \bmx_\tn{act}^c(\bmn,b) \rVert_2$ 
is minimized, while 
$F_\tn{ref} \equiv F_{m,i,j}^{n}$, 
$\bmx_\tn{act}^c(\bmn,b)$ is measured relative to the super cell 
$\Omega_{i,j}^s$,
and 
$\bmx_\tn{ref}^c \equiv \bmx_{m,i,j}^{c,s,n}$.
We refer the reader to Figure \ref{CMOFcentroid}.

The initial starting guess for finding the optimal MOF or CMOF
slope 
is determined from the optimal choice from the following (described
for the 3D case):
\begin{enumerate}
\item $(\Theta^{(1)},\Phi^{(1)})=\mbox{slope to angle}(\frac{\bmx_{ref}-\bmx_{uncapt}}{||\bmx_{ref}-\bmx_{uncapt}||})$
\item (first cut only) 
\begin{eqnarray}
(\Theta^{(2)},\Phi^{(2)})=\mbox{slope to angle}(\bmn^{(\mbox{CLSVOF})}).
\label{CLSVOFslope}
\end{eqnarray}
   $\bmn^{(\mbox{CLSVOF})}$ is derived from the CLSVOF 
   reconstructed slope \cite{sussman2000coupled}.
\item (first cut only) \\
   $(\Theta^{(3)},\Phi^{(3)})$ 
   is determined by way of the following 
   ``regression'' decision tree \cite{breiman1984classification}
   machine learning process
   (see Figure \ref{DT_figure}) :
   \begin{description}
   \item[(a)] at the very start of a simulation,
    randomly create 
    $n_{sample}=100^{3}$ 
    sample triplets for planes
    cutting through a given cell: 
    \begin{eqnarray*}
    (\Theta^{sample},\Phi^{sample},F^{sample}).  
    \end{eqnarray*}
   \item[(b)] Each sample is 
    associated with 
    $(\Theta^{key},\Phi^{key},F^{key})$ 
    which is determined as
    \begin{eqnarray*}
    (\Theta^{key},\Phi^{key})=
    \mbox{slope to angle}( \frac{\bmx_{ref,sample}-\bmx_{uncapt}}
       {||\bmx_{ref,sample}-\bmx_{uncapt}||} ) \\
    F^{key}=F^{sample}
    \end{eqnarray*}
   \item[(c)] Each associated triplet,
     $(\Theta^{key},\Phi^{key},F^{key})$ 
     with corresponding classification,
     \begin{eqnarray*}
     (\Theta^{classify},\Phi^{classify},F^{classify})\equiv
     (\Theta^{sample},\Phi^{sample},F^{sample})
     \end{eqnarray*}
     is added to the decision tree list.
   \item[(d)] After all of the sample data has been classified, 
     the decision tree list is split into two branches, 
     determined by the median in the critical
     splitting direction ($\Theta$, $\Phi$, or $F$).  
     See Figure \ref{DT_figure}.
     The critical direction maximizes the decrease in the
     classification variance between
     the parent and its associated two branches.
     (each branch having its own mean).  
   \item[(e)] The tree is split again into 4 branches, with the 
     next critical direction determined to maximize
     the classification variance reduction.  
     Note: the critical direction is the
     same for each branch on a given level.
   \item[(f)] the previous step is repeated recursively until 
     each branch in the tree contains just one piece of data.
     The number of levels in the tree cannot exceed
     $\mbox{ceiling}(\log_{2} n_{sample})$.
   \item[(g)] for the prediction phase, the decision tree is 
      traversed using the key,
      $(\Theta^{key},\Phi^{key},F^{key})$ in which
    \begin{eqnarray}
    (\Theta^{key},\Phi^{key})=
    \mbox{slope to angle}(\frac{\bmx_{ref}-\bmx_{uncapt}}
       {||\bmx_{ref}-\bmx_{uncapt}||}). \label{keydef} \\
    F^{key}=F_{ref} \nonumber
    \end{eqnarray}
    $(\Theta^{(3)},\Phi^{(3)})$ is the classification 
    obtained from traversing the decision tree using the key from
    (\ref{keydef}).  The maximum number of decisions to be 
    made in order to classify a key not contained in the training
    data set is 
    $\mbox{ceiling}(\log_{2} n_{sample})$.
   \end{description}
   Remarks:
   \begin{itemize}
   \item In 2D, at the very start of a simulation, one
    randomly creates $n_{sample}=100^{2}$ sample pairs for lines
    cutting through a given cell: 
    $(\Theta^{sample},F^{sample})$.  
   \item In RZ, a 2D decision tree is created corresponding 
    to each discrete value of $r_{i}=(i+1/2)\Delta x$.
   \item Analytical MOF reconstruction algorithms exist 
    \cite{milcent2020moment}, but not analytical CMOF
    reconstruction algorithms.
   \item for the CMOF reconstruction algorithm, $\bmx_{ref}$ 
    and $\bmx_{uncapt}$ are
    measured relative to the super cell 
    $\Omega_{i,j}^s$ 
    (\ref{eq:super_cell_def}).
    $\bmx_{uncapt}$ 
    is the centroid of the ``uncaptured'' 
    (i.e. ``unreconstructed'') region of
    $\Omega_{i,j}^s$.  For the first cut,  
    $\bmx_{uncapt}$ is the centroid of 
    $\Omega_{i,j}^s$ 
    (\ref{eq:super_cell_def}).
   \item if the super cell, $\Omega_{i,j}^s$, contains at most two
    materials, then our starting guess is guaranteed to lead to a 
    second order reconstruction, regardless of the number of ensuing 
    optimization iterations.  This is because the CLSVOF slope
    (\ref{CLSVOFslope}) by 
    itself leads to a second order method.  
   \item We have determined 
    anecdotally that a machine learning sample size of 
    $N_{1D}^{d}=100^{d}$
    where $d$ is the dimension of the optimization input data and 
    $N_{1D}$ represents a ``sample size per dimension,'' leads
    to errors comparable to iterating to convergence.  We have 
    arrived at this sample size by 
    studying the $N_{1D}$ dependence
    of the symmetric difference error for ``Zalesaks'' problem 
    (see Table \ref{zalesaktable}).
%% Zalesak's problem t=0, 24x24 maxlev=2: 1x1 cell size (after scaling), 
%% MOF
%% centroid error weighted by the 
%% volume fraction.
%% optimal error (Gauss Newton) 7.5E-5  (error x refvfrac)
%% optimal error (N=400^2 + CLSVOF guess) 1.8E-4  (error x refvfrac)
%% optimal error (N=200^2 + CLSVOF guess) 2.2E-4  (error x refvfrac)
%% optimal error (N=100^2 + CLSVOF guess) 2.7E-4  (error x refvfrac)
%% optimal error (N=50^2 + CLSVOF guess) 2.8E-4  (error x refvfrac)
%% CMOF
%% optimal error (Gauss Newton) 5.2E-3  (error x refvfrac)
%% optimal error (N=100^2 + CLSVOF guess) 5.2E-3  (error x refvfrac)
%% optimal error (N=50^2 + CLSVOF guess) 5.3E-3  (error x refvfrac)
%% optimal error (N=25^2 + CLSVOF guess) 5.3E-3  (error x refvfrac)
%%
%% 3D sphere t=0, 32^3: 1x1 cell size (after scaling), 
%% MOF
%% optimal error (Gauss Newton) 1.7E-5  (error x refvfrac)
%% optimal error (N=100^3 + CLSVOF guess) 3.2E-4  (error x refvfrac)
%% optimal error (N=50^3 + CLSVOF guess) 3.4E-4  (error x refvfrac)
%% optimal error (N=25^3 + CLSVOF guess) 3.5E-4  (error x refvfrac)
%% CMOF
%% optimal error (Gauss Newton) 1.1E-2  (error x refvfrac)
%% optimal error (N=100^3 + CLSVOF guess) 1.1E-2  (error x refvfrac)
%% optimal error (N=50^3 + CLSVOF guess) 1.1E-2  (error x refvfrac)
%% optimal error (N=25^3 + CLSVOF guess) 1.2E-2  (error x refvfrac)
   \end{itemize}
\end{enumerate}


\subsection{Reconstructing the distance function \label{MMRECON}}
Here we describe a distance function reconstruction algorithm that evaluate the exact signed distance function to the reconstructed interface.

Assuming that the reconstructed interface is available on the multimaterial cells, using a MOF or CMOF reconstruction, the algorithm first initialize the distance function to a large number with correct sign. Then in a narrow band the cells tagged for involvement in the redistancing procedure, that is, contributing information to and/or getting updated distance function value through this process. In the next steps, distances are evaluated from the center of an update cell to different possible interfacial points in its neighborhood, and material, $\phi_m$, and interface distance functions ,$\phi_{m1,m2}$, are updated consequently.

\begin{itemize}
\item Set all distance functions, $\phi_m$ and $\phi_{m1,m2}$, 
to a large negative value.
\item Iterate over all cells.  
  \begin{itemize}
  \item Using the reconstructed interface, change the sign of 
    $\phi_k$ and $\phi_{k,m2}$ to positive if material 
    $k$ is occupying the
    center of a cell.
  \item For a cell, check the neighbors for 
    occupying materials in a  $3 \times 3 \times 3$ 
    hypercube stencil around it. Count the
    materials $m$ present if $F_m>0.5$ in at 
    least one of the cells in the stencil, or if a 
    material level set changes sign between two 
    different cells in the stencil. If more than one 
    material are present in the stencil, this cell is 
    a \emph{support} cell for the redistancing 
    algorithm. Consequently, tag cells in the $9 \times 9 \times 9$ 
    hypercube around a support cell as \emph{update} cells.
 \end{itemize}
 \item Iterate over update cells.
   \begin{itemize}
   \item Iterate over the support cells in the $3 \times 3 \times 3$  
    hypercube around this update cell. 
     \begin{itemize}
     \item Evaluate distance to the corners, face centers, and 
      cell center in the support cell. This procedure gives 
      the exact distance if cell boundaries are 
      part of the material interface (Figure \ref{fig:redistancing}.a).
     \item Find the normal distance to each interface in the 
      support cell. Update the distance only if the 
      intersection point is within the support cell 
      (Figure \ref{fig:redistancing}.b).
     \item Find the intersection of each pair of the interfaces 
      in the support cell, and evaluate the distance 
      to the triple-point. Check the materials around the 
      intersection for updating the corresponding interface 
      distance functions (Figure \ref{fig:redistancing}.c).
     \item Find the intersection of the interfaces with 
      cell faces, evaluate the distance, and update the 
      corresponding distance functions (Figure \ref{fig:redistancing}.d).
     \item For 3D cases only, find the intersection line of 
       each pair of interfacial planes, and find the intersection 
      of the cell boundaries with the line. 
      Then, evaluate the distance to the intersection points, and update the corresponding distance functions (Figure \ref{fig:redistancing_3D}).
     \end{itemize}
   \end{itemize}
\end{itemize}

Notes:
\begin{itemize}
\item[-] In the redistancing algorithm, 
  after finding the distance to an interface or intersection of 
  interface with cell boundaries or other interfaces the 
  materials around the intersection points are found by 
  testing some points around the intersection to figure out 
  which material they belong to. These points are 
  picked by the combination of interface normal vectors 
  involved in the intersection points and are shown with white 
  diamonds in Figure \ref{fig:redistancing} and 
  Figure \ref{fig:redistancing_3D}.
\item[-] Both distance function value and normal vector is 
 updated if a distance measurement is found with smaller 
 magnitude compared to the stored value in $\phi_{m}$ and $\phi_{m1,m2}$.
\item[-] Applying the algorithm described above, an interface 
 distance function $\phi_{m1,m2}$ is not well defined when the 
 update cell belongs to neither material $m1$ nor 
 material $m2$ (Figure \ref{fig:redistancing_ils}.a). 
 An alternative measurement is used to extend the 
 interface distance functions (Figure \ref{fig:redistancing_ils}.b). 
 For a cell $\{i,j\}$ where $\phi_{m1,i,j}<0$ and $\phi_{m2,i,j}<0$ we have:
  \begin{eqnarray}
    \bmx_{m1} &=& \bmx_{i,j} - \phi_{m1,i,j} \bmn_{m1,i,j}, \\
    \bmx_{m2} &=& \bmx_{i,j} - \phi_{m2,i,j} \bmn_{m2,i,j}, \\
    \bmq &=& \bmx_{m1} - \bmx_{m2}.
  \end{eqnarray}
  \begin{equation}
    \left\{
      \begin{array}{lll}
        \phi_{m1,m2} = |\bmq|  &\tn{if } |\phi_{m1,i,j}| < |\phi_{m2,i,j}| \\
        \phi_{m1,m2} = -|\bmq| &\tn{if } |\phi_{m1,i,j}| \ge |\phi_{m2,i,j}| \\
      \end{array}
    \right.
  \end{equation}
\end{itemize}
 

\section{Numerical tests and results}


%%\biboptions{sort&compress}
\bibliography{cmof_paper}

\end{document}
%%% Local Variables:
%%% mode: latex
%%% TeX-master: t
%%% End:
